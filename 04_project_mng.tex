\chapter{Gestione Progetti}
\label{chap:pm}

%You may also put some code snippet (which is NOT float by default), eg: \cref{lst:random-code}.

%\lstinputlisting[float,language=Java,label={lst:random-code}]{listings/HelloWorld.java}

Come già anticipato, questo capitolo approfondisce tutte le attività che costituiscono il ciclo di vita
di un progetto di Peer Network, dalla fase di ideazione fino al rilascio al cliente e al supporto post-installazione.
L’analisi condotta ha permesso di individuare criticità lungo il processo e di elaborare proposte mirate alla loro risoluzione.

Nel corso del lavoro, è stato possibile seguire sia progetti già in corso che nuovi, partendo dalle riunioni
preliminari con i committenti per valutare l’opportunità di sviluppare un nuovo applicativo. Partecipare a incontri
con i clienti e con i team di progetto ha offerto una visione completa della gestione operativa e del lavoro
svolto dai diversi dipendenti, in particolare dai responsabili di progetto. Questa esperienza ha permesso di
sviluppare proposte concrete per migliorare i processi, adattandole alle reali esigenze dei team senza stravolgerne il metodo di lavoro.

L'analisi e le proposte sviluppate si basano costantemente sulla struttura del ciclo di vita del progetto proposta da Peer Network
e illustrata nella \Cref{sec:pm}.

\section{Analisi Stato}
I progetti per clienti esterni possono variare notevolmente in natura e, di conseguenza, nella durata che intercorre tra l’inizio e l'installazione:
\begin{itemize}
    \item aggiunta di funzionalità a un sistema esistente: da una a più settimane, a seconda della complessità;
    \item creazione di un nuovo applicativo con la maggior parte delle \ac{PBC} già presenti: da uno a diversi mesi;
    \item migrazione di un ERP aziendale (ad esempio, da SAP ECC a SAP S/4HANA) oppure creazione di un nuovo
    applicativo con numerose \ac{PBC} da sviluppare e/o logiche nuove: oltre tre mesi. I progetti di grandi dimensioni
    vengono suddivisi in sotto progetti, ciascuno della durata massima di tre o quattro mesi, per consentire rilasci graduali e stabilire le priorità.
\end{itemize}

Le proposte per un progetto possono provenire da diverse fonti. Nel caso di un nuovo cliente, il CEO di Peer Network verrà contattato direttamente.
Se il cliente ha già progetti o collaborazioni attive con l’azienda, la proposta può essere inviata via email o telefonicamente:
\begin{itemize}
    \item contattando il CEO;
    \item rivolgendosi al project manager di riferimento di un progetto in corso per la stessa azienda;
    \item contattando i responsabili del servizio di supporto. Questo è particolarmente utile quando inizialmente 
    il cliente ritiene di aver bisogno solo di una piccola miglioria in un'applicazione, ma dall'analisi della
    richiesta emerge la possibilità di sviluppare un nuovo progetto.
\end{itemize}

Un'ulteriore possibilità riguarda i progetti interni, nei quali non esiste un cliente esterno, ma è la Peer Network a essere cliente di sé stessa.

Nel caso di clienti esterni, tutte le proposte di progetto vengono sempre inoltrate al CEO, il quale gestisce l'intera fase iniziale del progetto.

    \subsection{Idea, Design, Economics}
    Sulla base di indicazioni formali e informali ricevute dai clienti, sia verbalmente che tramite email, il
    CEO elabora dei \textbf{mockup} utilizzando il software Axure\footnote{https://www.axure.com}. Se le funzionalità richieste sono già presenti in altri
    prodotti Peer Network, verranno presentate schermate simili a quelle utilizzate da altri clienti. In caso contrario,
    si esamineranno soluzioni personalizzate per soddisfare le esigenze specifiche.

    Dopo alcune \textbf{riunioni con i clienti}, si definiscono in modo più o meno dettagliato i \textbf{requisiti} richiesti. Questi ultimi,
    tuttavia, non vengono quasi mai redatti in modo collaborativo. Se la soluzione proposta al cliente è già stata
    implementata per altri, non si parte da zero nella raccolta dei requisiti. In base alle richieste ricevute, il CEO
    valuta quali \ac{PBC} possono essere utilizzate; nel caso in cui non siano disponibili, informa gli sviluppatori e avvia
    immediatamente il processo di creazione.

    Successivamente, il cliente comunica la disponibilità delle sue \textbf{risorse}, ovvero le macchine virtuali su cui Peer Network
    installerà gli ambienti di sviluppo, test e produzione. Inoltre, indica la data entro la quale desidera ricevere il progetto
    ed eventuali periodi in cui non sarà possibile lavorare.

    Una volta definiti i requisiti e le tempistiche per il completamento del progetto, il CEO redige un’\textbf{offerta} che il cliente
    accetta. Dopo aver preso in carico il progetto, egli crea uno spazio dedicato su Confluence per i \textbf{documenti interni}, dove
    inserisce un documento che riassume quanto concordato con il cliente. Ogni cliente dispone
    di un proprio spazio Confluence personalizzato e condiviso, contenente tutta la documentazione rilevante per i suoi progetti.
    In questa fase iniziale, il CEO si occupa di aggiornare lo spazio con il nuovo progetto.

    Il \textbf{Project Management Life Cycle} adottato si basa su un approccio Agile, combinato con un modello iterativo personalizzato
    ispirato a Kanban. Lo scopo del progetto è definito fin dall'inizio e rimane invariato. Le soluzioni vengono sviluppate in
    piccoli passi: le attività sono organizzate in base alla priorità per garantire il rilascio progressivo e frequente di moduli,
    cioè parti applicative funzionanti coerenti tra loro e autoconclusive. Al termine dello sviluppo e del testing di ciascun modulo,
    questo viene rilasciato al cliente  secondo le scadenze concordate. Idealmente, le persone responsabili indicate dal committente o
    gli utilizzatori del sistema dovrebbero eseguire test funzionali
    parallelamente al rilascio nell’ambiente di test, per verificare che i requisiti siano soddisfatti. Tuttavia, in molti casi, essi
    preferisco fornire feedback solo al completamento dell'intero lavoro.

    Il CEO è responsabile di questa fase, ricoprendo i ruoli di commerciale, analista e progettista. È lui a concordare con il cliente i requisiti, i costi e i tempi di realizzazione.

    Deliverables prodotti: mockup, documento richieste cliente (opzionale), offerta fatta al cliente.

    Ruoli interni coinvolti: CEO.

    Problemi riscontrati:
    \begin{itemize}
        \item mancano i verbali delle riunioni tenute con i clienti;
        \item non ci sono figure specifiche coinvolte, poiché le interazioni sono solo gestite dal CEO;
        \item frequentemente non viene stilato un elenco generico delle richieste, né tantomeno un insieme dettagliato
        di requisiti funzionali e non funzionali, quindi nelle fasi successivi ci si basa solo sui mockup e
        su informazioni riportate oralmente dal CEO;
        \item spesso non ci sono accordi definitivi con il cliente, il che porta a iniziare la fase successiva senza indicazioni
        chiare su budget e tempistiche. Questo comporta il rischio di dover effettuare un investimento interno;
        \item si utilizzano risorse sottratte dal lavoro operativo di sviluppo in altri progetti solo per chiedere consigli,
        fare screenshot di soluzioni già presenti presso i clienti e verificare funzionalità nelle \ac{PBC};
        \item spesso, il tempi e le scadenze stabilite con il cliente non sono sostenibili, dato l'alto numero di progetti in
        corso simultaneamente all'interno dell'azienda. Ciò aumenta il rischio di ritardi nelle consegne. Per evitare tali ritardi,
        tuttavia, si è costretti a ridurre significativamente la qualità del codice, della documentazione di gestione del progetto e tecnica,
        che risultano quasi assenti;
        \item non sono stati stabiliti deliverables standard da produrre in questa fase.
    \end{itemize}

    \subsection{Progetto Software}
    Per Peer Network, il project manager entra in gioco in questa fase, assumendo un ruolo cruciale nella riuscita del progetto.
    Questa figura professionale è responsabile della gestione ottimale delle risorse assegnate, del monitoraggio delle attività
    e del rispetto dei tempi e delle scadenze. Inoltre, egli coordina il team di progetto, che include sia risorse interne all'azienda
    sia quelle del cliente. Il project manager aggiorna costantemente lo spazio del cliente e quello interno all'azienda su Confluence,
    inserendo tutti i documenti che ritiene necessari.

        \subsubsection{Avvio}
        Il project manager avvia il processo di gestione del progetto analizzando le richieste ricevute dal CEO. Questo include la
        revisione dei mockup o presentazioni su Google Slides, insieme ad un eventuale documento riepilogativo delle esigenze del cliente.

        Successivamente, il project manager e il CEO organizzano una \textbf{piccola riunione} per presentare il nuovo progetto a un gruppo
        ristretto di persone potenzialmente coinvolte nel team. Questa riunione è raramente condotta in presenza dei clienti. A parte
        il project manager, gli altri sviluppatori hanno solo una comprensione generale delle aree di intervento, senza che venga designato un team leader o altri ruoli ufficiali.
        
        Il project manager è responsabile del \textbf{business case} del progetto, un foglio elettronico in cui viene inserita l'analisi dei \underline{costi}
        e dei \underline{ricavi pianificati}. Considerando il budget e i tempi concordati con il cliente nella fase iniziale, il compito del project manager
        è quello di trovare il modo migliore per rimanere all'interno di questi limiti. A tal fine, egli stima il numero di giorni/persona necessari
        per completare il lavoro e, utilizzando il costo orario di un dipendente, calcola il costo totale interno pianificato. Partendo da queste
        informazioni e dal budget, si può determinare quanto ricavo e margine so

        Deliverables prodotti: business case.

        Ruoli interni coinvolti: project manager, CEO.

        Problemi riscontrati:
        \begin{itemize}
            \item non esiste un metodo standard per affrontare questa fase, quindi ogni project manager la affronta a propria discrezione;
            \item la documentazione di progetto relativa a questa fase, in particolare quella relativa ai requisiti, viene redatta solo sporadicamente.
            Di conseguenza, nelle fasi successive il project manager è sempre costretto a rivolgersi al CEO per ottenere chiarimenti, anche da parte degli sviluppatori;
            \item spesso, non c’è un team di progetto definito, si coinvolgono gli sviluppatori disponibili o adatti ai compiti richiesti in base alle esigenze del momento;
            \item i ruoli all'interno del team non vengono quasi mai assegnati in modo formale;            
            \item non sono stati stabiliti deliverables standard da produrre in questa fase.
        \end{itemize}

        \subsubsection{Pianificazione}
        Partendo dall’idea e dalle funzionalità, a volte il project manager sviluppa una \textbf{\ac{WBS}}, scomponendo il progetto in
        numerosi elementi organizzati in una gerarchia chiara, simile a un diagramma ad albero. Dallo schema devono emergere chiaramente le
        funzionalità specifiche da sviluppare e i moduli in cui si intende dividere il progetto. Un modulo è definito come un insieme di funzionalità
        coerenti, che possono essere rilasciate in modo indipendente dalle altre, garantendo che ciascun modulo sia autoconclusivo. Idealmente, ogni
        volta che viene rilasciato un modulo, il cliente ha la possibilità di testarne le funzionalità. È fondamentale che la \ac{WBS} sia comprensibile
        anche per il cliente, pertanto, viene presentata a un livello logico non tecnico, mantenendo comunque il giusto grado di dettaglio.
        Nella Figura 2 è rappresentato un estratto della \ac{WBS} del progetto reale “esiSMART2”, elaborato dal project manager responsabile e
        presentato al cliente l'anno scorso. La radice rappresenta il modulo di riferimento, mentre gli altri elementi indicano le diverse
        funzionalità richieste, con un livello di dettaglio sempre crescente.

        %figura WBS

        È successo raramente che si effettuasse un'\textbf{analisi dei rischi}, la quale veniva condotta dopo aver scomposto il problema in moduli
        e relative funzionalità. Questo approccio è stato adottato in progetti in cui si prevedevano significative criticità in alcuni moduli.
        A seguito dell'analisi, si valutava come mitigare o risolvere il problema, riportando le informazioni al cliente e condividendo una possibile strategia.

        Per avere una visione complessiva del progetto, il project manager elabora un \textbf{Gantt}, posizionando tutti i moduli da sviluppare all'interno di un
        calendario. Questo processo tiene conto delle scadenze concordate con il cliente e delle dipendenze tra i vari moduli. Non tutti i project manager
        seguono questa prassi; coloro che lo fanno, utilizzano Gantt Pro. Talvolta, prima di redigere il Gantt, si crea uno \textbf{schema generale di pianificazione
        rilasci} fino al lancio, privo di date, ma che evidenzia l’ordine dei rilasci (come illustrato nella Figura 3). In entrambi i casi, la pianificazione
        viene sottoposta a validazione da parte del cliente.

        %figura rilasci

        Per ogni progetto, il project manager o il CEO crea un \textbf{progetto su Jira}, inserendo le informazioni nella dashboard Timeline:
        \begin{itemize}
            \item Partendo dalla \ac{WBS}, i moduli identificati scomponendo le funzionalità richieste vengono inseriti come \underline{epic}. Questi rappresentano
            delle milestone di progetto e al loro completamento vengono rilasciati. A causa del loro livello generale, una epic non può essere
            assegnato a una persona specifica, il che significa che nessuno sviluppatore può indicare il tempo impiegato per il suo sviluppo. Ad
            esempio, come illustrato nella Figura 2, una epic potrebbe essere “R11 Gestione documenti Standard e PDF“, dove R11 indica che sarà l’undicesimo modulo da rilasciare.

            \item Ogni epic è ulteriormente scomposta in:
                \begin{itemize}
                    \item \underline{user stories} (opzionale): rappresentano la descrizione di una funzionalità e sono formulate in una semplice frase seguendo
                    il modello standard ruolo-goal-beneficio del framework Agile Scrum. A causa del loro livello di generalità, non possono essere 
                    assegnate a una persona specifica, rendendo impossibile per gli sviluppatori stimare il tempo necessario per il loro sviluppo. Le 
                    user stories non sono sempre presenti, poiché la loro necessità dipende dalla natura del progetto e, se ritenute utili, vengono 
                    redatte durante l’analisi funzionale e assegnate alle rispettive epic. Da queste si derivano i task, ovvero le attività necessarie per soddisfare le richieste.

                    \item \underline{task}: sono le attività da svolgere che, pur essendo di un livello non tecnico, rappresentano l'ultimo tipo di elemento
                    comprensibile per il cliente. Facilitano anche la comunicazione con quest'ultimo e hanno già un carattere operativo. Ogni task
                    viene assegnato a una persona responsabile del suo sviluppo e la gestione di queste attività è compito del project manager. Ogni
                    dipendente al quale viene assegnato un task può caricare del tempo impiegato per lavorare a questa attività. Ad esempio, come mostrato
                    nella Figura 1, per ogni foglia viene creato un task con un titolo descrittivo, che fa riferimento anche al nodo padre.
                \end{itemize}
            
            \item Ogni task è suddiviso in \underline{subtask}, che rappresentano attività tecniche non rilevanti per il cliente, poiché si trovano a un livello
            implementativo. Anche questi compiti vengono assegnati a una persona responsabile dello sviluppo. Idealmente, il team leader, ovvero il
            responsabile del team di sviluppo incaricato degli aspetti tecnici, dovrebbe occuparsene. Tuttavia, in molte occasioni, non viene nemmeno
            nominato e quindi anche questo compito ricade sul project manager. I dipendenti possono caricare le ore di lavoro su ogni subtask.
        \end{itemize}

        Per ogni elemento inserito nel progetto Jira è buona prassi fornire descrizioni testuali, immagini e link a documenti pertinenti
        (sul progetto Confluence associato), così da garantire spiegazioni più dettagliate.
        Molto spesso, le epic e/o i task non vengono tutti scomposti inizialmente rispettivamente in task e subtask, seguendo un approccio iterativo.
        Questo implica che i primi elementi da sviluppare verranno scomposti con la maggiore granularità possibile fin da subito. Gli altri, invece,
        saranno affinati progressivamente durante il progetto, con un preavviso adeguato, quando i precedenti sono in fase di sviluppo.
        Nella Figura 4 è mostrato un esempio di progetto su Jira. Nella Timeline, le epic sono rappresentate da icone viola. Ogni epic include task
        contrassegnati da un'icona azzurra con una spunta. Nella finestra a destra, è visibile il dettaglio del task, completo di spiegazioni e immagini,
        insieme ai subtask, che sono indicati come “child issues” e un'icona azzurra.
        
        %figura

        Per quanto riguarda le \textbf{risorse umane} assegnate a task e subtask, è importante notare che, man mano che le attività si allontanano nel tempo,
        aumenta l'incertezza riguardo alle persone specifiche che si occuperanno di esse. Tuttavia, per una prima suddivisione, si assegnano risorse
        generiche, come ad esempio frontend developer, ABAP developer o backend developer.
        
        Una volta completata la suddivisione, si passa alla pianificazione sulla Team Board di Jira, organizzando le attività a lungo termine.
        In questa fase, si stabilisce l'ordine di sviluppo delle epic e dei task, insieme alla loro durata, seguendo lo schema Gantt definito inizialmente.
        A seconda delle dimensioni del progetto, si utilizzano mesi o settimane come unità di misura. Questa fase è cruciale per comprendere se nel lungo
        periodo ci sarà bisogno di risorse umane con competenze specifiche. Ad esempio, se tra due mesi saranno necessari cinque sviluppatori ABAP,
        è fondamentale verificare se ci sono risorse interne disponibili in quel periodo per occuparsi di questo progetto. In caso contrario, sarà necessario cercare e assumere nuovi dipendenti in tempo utile.

        Deliverables prodotti: \ac{WBS} (opzionale), Gantt (opzionale), schema di pianificazione rilasci (opzionale), analisi rischi (opzionale), pianificazione.

        Ruoli interni coinvolti: project manager, team leader (opzionale).

        Problemi riscontrati:
        \begin{itemize}
            \item non esiste un template standard della WBS;
            \item si riscontra difficoltà nel capire come scomporre e con quale granularità l’idea iniziale in una WBS completa;            
            \item la maggior parte delle volte, i project manager non fanno la WBS e scompongono il problema direttamente sul relativo progetto Jira;            
            \item non tutti i project manager utilizzano il diagramma di Gantt;            
            \item nelle epic, user stories, task e subtask non sempre vengono fornite descrizioni o link a documenti pertinenti con spiegazioni più dettagliate;            
            \item esiste una notevole confusione riguardo all'organizzazione di un progetto su Jira in epic, task e subtask e
            ogni project manager adotta un approccio diverso dagli altri;           
            \item non c'è coerenza nella suddivisione dei task, il che porta a progetti privi di subtask o che ne contengono solo alcuni;            
            \item non esiste un team ben definito, quindi una persona potrebbe avere task o subtask assegnati su tutti i progetti aziendali nella stessa settimana;            
            \item poiché non ci sono team definiti, anche i ruoli non sono specificati, come un project manager potrebbe trovarsi a svolgere attività di sviluppo in un altro progetto.
            Di conseguenza, le persone spesso faticano a comportarsi secondo il ruolo necessario in quel momento;
            \item i project manager, essendo anche figure tecniche, spesso non si concentrano su una pianificazione di alto livello
            dettagliata, ma tendono a passare direttamente allo sviluppo;            
            \item la pianificazione risulta complessa, poiché l’approccio degli ingegneri è quello di non fornire stime su aspetti incerti, come la previsione
            dei tempi per attività future;        
            \item i subtask sono frequentemente gestiti dal project manager, sia per l'assenza di un team leader sia perché non vuole
            delegare e si occupa direttamente delle attività;
            \item non sono stati stabiliti deliverables standard da produrre in questa fase.
        \end{itemize}

        \subsubsection{Esecuzione}
        La \textbf{programmazione} è l’organizzazione delle attività su un orizzonte temporale breve. Questa viene fatta settimanalmente dal project manager assegnando
        task e subtask raffinati a delle persone fisiche. Ogni venerdì il responsabile tecnico dell’azienda controlla che nella settimana successiva tutti i
        programmatori siano occupati in attività e che tutti i task e subtask programmati per quel periodo siano assegnati a persone reali (non siano più
        assegnate a risorse generiche). Nella Figura 5 c'è una visione del Gantt del plugin Team Board su Jira, dove si vede la programmazione effettiva.
        Visto che ad ogni persona vengono assegnati più task e subtask da sviluppare, è suo compito registrare le ore di lavoro che impiega per ciascuna,
        oltre a spiegare quali attività ha svolto e aggiornarne lo stato di avanzamento (da completare, in corso, testato, completato).

        %figura 
        Durante lo \textbf{sviluppo}, i dipendenti caricano le modifiche al codice nel repository di progetto su Bitbucket, utilizzando le funzionalità del sistema
        di controllo di versione Git. Grazie all’integrazione tra gli strumenti Atlassian, è sufficiente adottare una convenzione standard per i nomi dei
        branch e dei commit affinché questi vengano associati automaticamente ai rispettivi task o subtask di Jira. Questo meccanismo permette di semplificare
        le attività di controllo e verifica. I commit devono seguire il formato \texttt{<codice Jira del task o subtask> <tipo di commit>: <descrizione>} (la
        seconda parte segue il Conventional Commit\footnote{https://www.conventionalcommits.org/en/v1.0.0/}), mentre per i branch si utilizza il formato \texttt{<codice Jira del task o subtask>-<descrizione>}.

        Il \textbf{rilascio del software} avviene esclusivamente in modo manuale, poiché non sono presenti meccanismi automatizzati come \textbf{continuous integration}
        o \textbf{continuous delivery}. Questa modalità dipende dall'attuale configurazione, in cui le repository di progetto risiedono su Bitbucket, una piattaforma
        cloud, e l’accesso alle macchine virtuali dei clienti richiede l’utilizzo della VPN. Al momento, non sono state individuate soluzioni per superare le limitazioni
        legate all'accesso alle macchine virtuali tramite Bitbucket.

        Per quanto riguarda il \textbf{workflow di rilasci e test prima del rilascio in produzione}, il processo inizia nell'\underline{ambiente di sviluppo}, noto come DEV.
        Sebbene sia buona prassi eseguire gli Unit Test in questo ambiente, in realtà vengono svolti raramente a causa della mancanza dei dati necessari negli \ac{ERP},
        rendendo difficile condurre test significativi. Quando le funzionalità sono considerate stabili, si procede con il rilascio nell'\underline{ambiente di test} denominato
        QUALITY. In questa fase, è possibile rilasciare in modo indipendente componenti di frontend, \ac{ESI} o backend. Nel caso in cui non sia possibile eseguire Unit
        Test su DEV, questi vengono effettuati in QUALITY. In questo ambiente, gli sviluppatori eseguono anche Use Case
        Test e test funzionali, verificando manualmente che tutto funzioni correttamente in base alle user stories identificate in precedenza o alle richieste del cliente.
        Inoltre, in QUALITY i clienti dovrebbero eseguire gli User Acceptance Test man mano che le funzionalità vengono rilasciate, per assicurarsi che quanto sviluppato
        risponda alle loro esigenze. Qualora i committenti effettuino i loro controlli, possono inviare i riscontri tramite email, telefonata o attraverso il service
        management di Jira. Tuttavia, non tutti i clienti sono disposti a testare progressivamente, poiché alcuni preferiscono che i cambiamenti vengano direttamente
        trasferiti in PROD, fornendo riscontri solo al termine del lavoro. Se tutto risulta stabile e accettato, i cambiamenti vengono trasferiti nell'ambiente di produzione,
        come descritto nella fase successiva. In alcuni casi, ci sono
        committenti che non dispongono di un ambiente di QUALITY, pertanto il rilascio avviene direttamente da DEV a PROD.

        In merito al \textbf{monitoraggio delle attività operative}, il project manager effettua una verifica settimanale dello stato di avanzamento del progetto utilizzando Jira.
        In alternativa, può consultare il team leader del progetto, se disponibile, o interagire direttamente con gli sviluppatori. Non è prevista una riunione fissa e gli
        aggiornamenti vengono forniti verbalmente. In presenza di \underline{problemi gravi}, il team leader, uno sviluppatore esperto o il project manager prepara un
        documento tecnico contenente una spiegazione dettagliata e le possibili soluzioni. Il project manager lo revisiona e lo adatta per renderlo comprensibile
        al cliente. Successivamente, il cliente viene contattato telefonicamente o via email con il documento allegato e si organizza una videochiamata per
        presentare il problema e le soluzioni proposte dal team leader. Durante l'incontro, si concorda insieme al cliente la linea d'azione da seguire.
        Per \underline{problemi minori}, come uno sviluppatore che non comprende il task assegnatogli, non viene redatto un documento, ma si discute direttamente per
        risolvere la questione. Se si tratta di un problema tecnico, si coinvolge un esperto per ricevere assistenza. Nel caso di un \underline{problema} che può essere 
        \underline{interessante per tutti}, viene
        creato un documento su Confluence che spiega il problema e la relativa soluzione. Questo documento sarà accessibile a chiunque, così che possa essere utilizzato quando si presenterà un problema simile.
        In ogni caso, c'è sempre una collaborazione attiva tra tutti i dipendenti dell’azienda, in caso di necessità.

        All'inizio di progetti di grandi dimensioni o in quelli che coinvolgono nuovi membri, si tiene al mattino uno \textbf{stand up meeting} informale di dieci minuti.
        Durante questa breve riunione, i partecipanti discutono i problemi emersi il giorno precedente e possono richiedere supporto per quella giornata. Alcuni
        progetti si prestano naturalmente a questo tipo di incontro, mentre altri potrebbero non essere adatti. La necessità di un stand up meeting dipende anche
        dalla dinamica del team: se i membri sono molto affiatati, esiste già un'intesa e un metodo di lavoro consolidato. Inoltre, la durata del progetto influisce
        sulla necessità di questa breve riunione, poiché per progetti di qualche settimana, un incontro quotidiano potrebbe non essere necessario.

        Per un efficace \textbf{monitoraggio economico} del progetto, è fondamentale controllare sistematicamente il business case su base mensile.
        Questo implica un'analisi dei costi e dei ricavi pianificati, stabiliti all'inizio del progetto, confrontati con i \underline{costi e i ricavi effettivi}.
        I valori reali si aggiornano progressivamente, in base alle ore impiegate dalle persone nello sviluppo del progetto. 
        Il project manager ha il compito di monitorare costantemente questi quattro valori, al fine di anticipare eventuali criticità, poiché il suo
        obiettivo è massimizzare il \underline{margine}. Se il progetto si estende all'anno successivo, ai project manager viene richiesto di elaborare una previsione
        dei costi e dei ricavi anche per l'anno successivo, per avere un'idea chiara delle aspettative. 
        Ogni mese dell'anno nuovo, i costi e i ricavi effettivi vengono aggiornati per quel mese, mentre i costi e i ricavi pianificati per il mese successivo
        vengono rivisti in base ai dati effettivi del mese corrente, poiché questi ultimi sono considerati più affidabili.

        La frequenza degli \textbf{allineamenti con il cliente} è una scelta concordata ad inizio progetto e modificabile durante la sua esecuzione. Di solito,
        il project manager organizza incontri settimanali con il cliente. Per progetti di minore entità, come piccole modifiche a sistemi esistenti, si decide
        settimanalmente con il cliente quali sviluppi intraprendere e rilasciare in quella settimana. Per progetti a lungo termine, già ben pianificati, gli
        allineamenti con il cliente avvengono almeno ogni due mesi. Alcuni project manager documentano sempre questi incontri e li pubblicano nello spazio Confluence del cliente.

        Il project manager è responsabile della pubblicazione di vari documenti nello spazio Confluence del cliente, selezionando quelli che possono risultare
        utili in base alle esigenze specifiche del progetto. Tra i \textbf{documenti condivisi} possono figurare documentazione tecnica richiesta, problemi riscontrati e le soluzioni adottate.

        Deliverables prodotti: documentazione tecnica (su richiesta), documentazione codice, business case aggiornato, verbali allineamenti cliente (opzionale).

        Ruoli interni coinvolti: project manager, sviluppatori, team leader (opzionale).

        Problemi riscontrati:
        \begin{itemize}
            \item project manager e responsabili non hanno una visione chiara dello stato generale del progetto, poiché non tutti i dipendenti aggiornano in modo corretto e tempestivo
            lo stato di avanzamento dei task e subtask a loro assegnati;
            \item c'è una scarsa propensione alla documentazione su Confluence riguardo a quanto realizzato;            
            \item non tutti i project manager verbalizzano gli allineamenti con il cliente, perndendo informazioni preziose;
            \item non è stata definita una lista di documenti importanti da pubblicare sistematicamente per ogni progetto nello spazio Confluence del cliente;
            \item assenza di meccanismi per il rilascio e i test automatici;
            \item non sono stati stabiliti deliverables standard da produrre in questa fase.
        \end{itemize}

        \subsubsection{Rilascio}
        La frequenza dei \textbf{rilasci in produzione} viene concordata con i clienti all'inizio del progetto, ma può essere modificata anche durante lo sviluppo.
        I tipi di rilascio standard sono i seguenti:
        \begin{itemize}
            \item settimanali: per moduli di piccole dimensioni;
            \item bisettimanali: per moduli che richiedono un'analisi aggiuntiva;
            \item mensili: per moduli che necessitano di una quantità significativa di test.
        \end{itemize}  
        
        In produzione è possibile rilasciare solo un modulo completo, comprendente di frontend, \ac{ESI} e backend correlati, affinché sia completamente funzionante.
        All'inizio di un nuovo progetto, il primo componente ad essere caricato in produzione è \ac{ESI} contenente le \ac{PBC} core, poiché sono stabili e rappresentano
        funzionalità di utilità base necessarie per il corretto funzionamento generale.
        Il rilascio del software avviene esclusivamente in modo manuale, come descritto nella fase precedente.
        
        Generalmente, il \textbf{termine di un progetto} non è gestito in modo convenzionale, poiché non presenta una vera e propria conclusione. Infatti, c'è un continuo
        supporto e servizio di assistenza, come spiegato meglio nella fase successiva. Un progetto si considera concluso solo quando l'azienda cliente smette di utilizzare l'applicazione.        
        In un progetto di migrazione, la conclusione coincide con il completamento del passaggio al nuovo sistema.
        In questo contesto di migrazione, viene organizzata una \textbf{retrospettiva
        interna} dedicata principalmente al team di progetto. La \textbf{retrospettiva con il cliente}, invece, viene effettuata solo su richiesta del cliente stesso o quando è
        necessario affrontare aspetti critici, un’esigenza più comune nei progetti di grandi dimensioni.
        La retrospettiva è guidata dal project manager e condotta dal team leader insieme all’intero team. Durante l’incontro, si analizzano e si confrontano le esperienze
        vissute durante il progetto, favorendo uno scambio di opinioni costruttivo.
        Nei progetti che comportano cambiamenti evolutivi su sistemi esistenti, non si prevedono retrospettive, poiché la loro durata è generalmente troppo breve per giustificare tale attività.
        
        Deliverables prodotti: soluzione prodotta, retrospettiva interna (opzionale), retrospettiva cliente (opzionale).

        Ruoli interni coinvolti: project manager, team leader (opzionale), sviluppatori.

        Problemi riscontrati:
        \begin{itemize}
            \item assenza di meccanismi automatici per il rilascio e i test;
            \item mancanza di un documento di rilascio finale;            
            \item la riunione di retrospettiva viene fatta sporadicamente e solo per i progetti in cui ci sono stati aspetti negativi da segnalare;
            \item non sono stati stabiliti deliverables standard da produrre in questa fase.
        \end{itemize}

    \subsection{Supporto e Servizio}
    Dopo circa uno o due mesi dalla consegna ufficiale del sistema al cliente, quando è installato e considerato stabile, avviene il trasferimento della
    responsabilità del progetto al team di supporto. Per questo motivo, per ogni cliente viene creato un progetto su Jira per l'Application Management Service (AMS),
    così che il team di supporto possa indicare in dettaglio tutte le attività svolte, nonché pianificarle e registrare le ore lavorate.

    I clienti possono comunicare con il team attraverso la segnalazione di problemi (noti come ticket) nel Jira Service Management, un ecosistema progettato per
    guidarli nella corretta formulazione delle richieste di supporto, manutenzione o evoluzione. In generale, per ogni segnalazione viene effettuata un'analisi
    del problema riportato e, in base alle richieste, vengono intraprese attività di correzione o sviluppo. 

    Il team di supporto si occupa di diverse attività, tra cui:
    \begin{itemize}
        \item \textbf{ricezione e classificazione delle segnalazioni}: il team di supporto controlla quotidianamente le segnalazioni ricevute. Una volta ricevuti
        i ticket di richiesta di supporto, un membro del team analizza la richiesta per verificare che il cliente abbia correttamente indicato nel ticket gravità,
        priorità e tipologia di problema. Il team stima le ore o i giorni necessari per le attività da svolgere per ogni ticket mentre analizza il problema riportato. Si identificano tre tipi di ticket: 
            \begin{itemize}
                \item malfunzionamento: viene riscontrato un problema nel sistema, quindi è necessaria una manutenzione correttiva;
                \item incident: il sistema non lavora correttamente e, di conseguenza, il cliente sta perdendo lavoro e denaro;
                \item cambiamenti: modifiche o aggiunta di funzionalità, quindi richieste evolutive.
            \end{itemize}
        \item \textbf{assistenza e supporto al cliente}: il team risponde a qualsiasi dubbio o richiesta del cliente, anche quando quest'ultimo non ricorda come eseguire una determinata operazione;
        \item \textbf{attività di manutenzione correttiva}: nel caso di malfunzionamenti o incident, normalmente le modifiche da fare sono relativamente piccole;
        \item \textbf{gestione delle richieste evolutive}: quando viene richiesto un cambiamento o l'aggiunta di funzionalità, se le attività necessarie hanno
        una durata stimata inferiore ai due giorni, il team di supporto se ne occupa direttamente. Altrimenti, la richiesta viene inoltrata ad un project manager di
        riferimento del progetto di sviluppo, il quale si occuperà di assegnare il lavoro agli sviluppatori. A seconda della complessità delle modifiche, il responsabile
        può decidere di organizzare un nuovo progetto con un team di sviluppo dedicato.        
    \end{itemize}

    Il team di supporto si occupa di integrare la \textbf{documentazione tecnica} relativa a ciascun progetto su Confluence, riportando tutte le modifiche apportate.
    Qualora alcune soluzioni siano state implementate in modo tale da risultare utili anche per altri progetti, queste vengono dettagliate in uno spazio di Confluence
    interno all’azienda, accessibile a tutti i dipendenti. Tuttavia, queste informazioni vengono raramente condivise con il cliente, poiché si tratta di aspetti
    tecnici che non sono di suo interesse. Inoltre, poiché in molti progetti la documentazione generale è carente, non è opportuno condividere la parte relativa alla
    manutenzione, in quanto risulta incompleta per definizione.

    Deliverables prodotti: documentazione interna del progetto aggiornata.

    Ruoli interni coinvolti: team di supporto.

    Problemi riscontrati:
    \begin{itemize}
        \item il team si trova in difficoltà nel comprendere dove e come intervenire, a causa di:
            \begin{itemize}
                \item la documentazione del codice (Javadoc) è quasi inesistente e il codice stesso è poco commentato (spiegazione di logiche complesse);
                \item la documentazione di progetto è caotica e insufficiente.
            \end{itemize}
        \item ci sono numerosi ritardi nelle consegne, poiché spesso si stima che la risoluzione di un task richieda poco tempo, ma poi si scopre codice poco manutenibile e debito tecnico:
            \begin{itemize}
                \item il sistema del cliente è obsoleto e non è stato aggiornato in concomitanza con le modifiche non retrocompatibili effettuate nel corso degli anni;
                \item il codice del progetto è spesso di scarsa qualità, perché chi lo ha scritto non aveva esperienza e non è stato controllato durante lo sviluppo.
            \end{itemize}
        \item non esistono linee guida chiare sulla struttura della documentazione del servizio di supporto;
        \item non sono stati stabiliti deliverables standard da produrre in questa fase.
    \end{itemize}

\section{Soluzioni Proposte}

    \subsection{Idea, Design, Economics}

    \subsection{Progetto Software}
        \subsubsection{Avvio}
        \subsubsection{Pianificazione}
        \subsubsection{Esecuzione}
        \subsubsection{Rilascio}

    \subsection{Supporto e Servizio}
