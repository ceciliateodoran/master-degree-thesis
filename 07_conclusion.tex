\chapter{Conclusioni e Lavori Futuri}

%(1) Briefly recall problem, starting point and methods adopted,
%(2) Briefly report Findings,
%(3) Briefly discuss benefits/limitations,
%(4) Discuss Future Work

Lo scopo del lavoro svolto è stato ottimizzare la gestione dei progetti aziendali della \textit{Peer Network}.
Per proporre soluzioni concrete, introducendo nuove attività o modificandone alcune esistenti, è stato
innanzitutto necessario analizzare a fondo il processo attuale, dall’iniziale contatto con il cliente
fino all’installazione completa della soluzione e al successivo servizio di supporto. In un secondo momento,
attraverso un confronto con la direzione e i project manager, sono state selezionate alcune proposte
da implementare. Parallelamente, sono state svolte attività di sviluppo nell’applicativo interno \ac{PAM}.
Oltre a piccoli interventi sul frontend, il contributo principale ha riguardato la realizzazione
dell’interfaccia grafica, con la relativa logica, per il riepilogo dei dati economici di ciascun
progetto, consentendo la visualizzazione di costi, ricavi e margini sia pianificati che effettivi.

Per valutare l’efficacia delle soluzioni adottate e il loro impatto in termini di risparmio
di tempo e miglioramento della qualità del lavoro svolto, è essenziale che l’azienda le applichi in
modo sistematico. Solo dopo alcuni mesi, una volta consolidate come prassi operative per tutti i dipendenti,
sarà possibile effettuare un confronto significativo con i progetti precedenti.
Poiché al momento non è possibile un’analisi comparativa diretta, data la necessità di un periodo di
assestamento, un’alternativa potrebbe essere il monitoraggio progressivo delle nuove metodologie.
In particolare, misurare il tempo impiegato dalle risorse per svolgere specifiche attività prima e
dopo l’introduzione delle nuove pratiche permetterebbe di raccogliere dati concreti sul loro impatto.

Per quanto riguarda \ac{PAM}, un’evoluzione naturale del sistema consisterebbe nell’integrare al suo
interno tutte le funzionalità attualmente gestite tramite fogli elettronici o altri strumenti esterni,
trasformandolo a tutti gli effetti in un sistema informativo aziendale.
Partendo dall’interfaccia sviluppata, focalizzata sulla gestione dei dati economici di progetto, si
potrebbe ampliare il sistema includendo una sezione dedicata al monitoraggio di ricavi e costi,
sia pianificati che reali, su base mensile. I dati previsionali, solitamente inseriti dal project manager
in un foglio elettronico, verrebbero direttamente registrati nell’applicativo, consentendo l’introduzione
di strumenti avanzati di analisi, come grafici e statistiche sulle proiezioni finanziarie.
Una volta integrati tutti questi elementi, si potrebbe sviluppare un’area riservata alla direzione e
all’amministrazione aziendale, dedicata all’analisi economica complessiva. Questa sezione offrirebbe
statistiche mensili e annuali su ricavi, costi e margini aziendali, sintetizzando le informazioni già
in possesso. Un simile strumento risulterebbe estremamente utile per monitorare l’andamento economico
della \textit{Peer Network} e supportare decisioni strategiche su investimenti in nuove risorse,
attrezzature e aggiornamenti.