\chapter{Introduzione}
\label{chap:introduction}
\textit{Peer Network} è una piccola azienda ravennate specializzata nello sviluppo di soluzioni 
per la re-ingegnerizzazione e la digitalizzazione dei processi aziendali. Il numero di dipendenti 
è raddoppiato negli ultimi anni e con questo incremento è stato necessario introdurre delle regole 
operative e una chiara suddivisione dei ruoli, così da migliorare l'organizzazione e, di conseguenza, 
l'efficienza del lavoro.

Grazie all'aumento della forza lavoro, è cresciuto parallelamente il numero di progetti richiesti 
dai clienti, sia da nuovi committenti che da quelli con una collaborazione già in essere. Poiché 
i sistemi sviluppati dall’azienda si integrano online con gli \textbf{\ac{ERP}}\footnote{Un 
\ac{ERP} è una suite di moduli applicativi che coprono l'intera gamma dei processi aziendali, 
consentendo l'integrazione tra diversi sistemi informativi. I moduli possono gestire molte aree, 
come ad esempio logistica, finanza, risorse umane, produzione e vendite. Questo software si basa 
sui principi di unicità e circolarità dell'informazione, estendibilità e modularità funzionale, 
oltre ad essere costruito secondo le best practices del settore di riferimento.}
dei clienti e la maggior parte dei processi aziendali presenta caratteristiche simili, è stata 
presa la decisione di ottimizzare le funzionalità sviluppate creando insiemi di funzionalità 
interconnesse, noti come \textbf{\ac{PBC}}. Questi moduli, componibili in vari modi, possono essere utilizzati 
per ampliare sistemi esistenti o per svilupparne di nuovi. \textit{Peer Network} ha sviluppato la 
piattaforma di integrazione \textbf{\ac{ESI}} per ospitare tutte le \ac{PBC}. Grazie alla loro flessibilità, 
le soluzioni applicative offerte ai clienti prendono il nome di \textbf{\ac{SBS}}.

In seguito a questi cambiamenti, l’azienda ha iniziato ad introdurre strategie per coordinare tutte 
le attività relative ai progetti dei clienti e per proseguire nella creazione di nuove \ac{PBC}.

Un'altra iniziativa interna introdotta nell'ultimo anno è lo sviluppo di una applicazione per supportare 
il lavoro dell'amministrazione, chiamata \textbf{\ac{PAM}}. Tra le funzionalità già operative vi è il monitoraggio 
mensile delle ore di lavoro dei dipendenti all'interno dei progetti, oltre alla creazione di report 
mensili da inviare ai clienti.

Con l’introduzione di questo ulteriore progetto e la necessità di distribuire il lavoro tra risorse limitate, 
è emerso che le regole organizzative adottate nel tempo non erano più sufficienti per gestire contemporaneamente 
tutti i progetti. Pertanto, \textit{Peer Network} ha la necessità di analizzare l’intero processo di gestione 
dei progetti aziendali, così da identificare tecniche di miglioramento. Inoltre, è richiesto lo sviluppo di 
nuove funzionalità per il \ac{PAM}, oltre ad un approfondimento sulla sua architettura software, poiché 
anch'essa si basa sulle \ac{PBC}.

Nei prossimi capitoli verranno illustrati gli strumenti adottati, inclusi i linguaggi di programmazione e 
i framework. Successivamente, sarà analizzata la gestione dei progetti in uso nel novembre del 2024 e 
verranno presentate alcune proposte per migliorarla. Inoltre, dopo un’analisi tecnica di \ac{PAM}, saranno 
descritte le nuove funzionalità software sviluppate. Infine, verranno illustrati i risultati ottenuti, 
con conclusioni e spunti per sviluppi futuri dell'azienda.
