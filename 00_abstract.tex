\begin{abstract}	
    %Very brief (e.g. 250-300 words)
    %Context, Problem/Objectives, Methods/Contribution, Results, Conclusions

Il presente lavoro nasce dall’esigenza di \textit{Peer Network} di ottimizzare la gestione
dei progetti aziendali. L’azienda opera nel settore della digitalizzazione e re-ingegnerizzazione
dei processi per i propri clienti, sviluppando applicazioni componibili anziché software interamente
personalizzati. Ogni entità di business coinvolta viene implementata come un'unità modulare,
denominata \acl{PBC}, che integra sia i propri dati che le azioni eseguibili su di essa. Oltre a
realizzare soluzioni per i clienti con questo approccio, \textit{Peer Network} sviluppa
nuove \acl{PBC} e un’applicazione gestionale interna, \acl{PAM}, volta a migliorare l’efficienza
delle proprie attività amministrative e organizzative.

L’analisi è iniziata esaminando nel dettaglio tutte le fasi del ciclo di vita di un progetto, dalle
attività gestionali e operative iniziali, fino all’installazione del sistema e alla fase di supporto.
Successivamente, partendo dai principi teorici del \acl{PMBOK}, dall’esperienza maturata in altri contesti
lavorativi e considerando la necessità di non poter stravolgere il metodo di lavoro aziendale, sono state
proposte soluzioni pratiche per rendere più efficiente la gestione dei progetti.

Una delle soluzioni concretamente implementate durante questo lavoro riguarda l’automazione
dell’inserimento e dell’aggiornamento dei dati economici di ciascun progetto, inclusi costi, ricavi e
margini, sia pianificati che effettivi. In precedenza, questa gestione avveniva manualmente tramite
fogli elettronici, con un dispendio di tempo e un maggior rischio di errore. L’integrazione di
queste funzionalità in \acl{PAM}
consente invece di ridurre i tempi di compilazione, centralizzare i dati in un’unica piattaforma e migliorare
l'affidabilità delle informazioni grazie all’automazione di alcuni inserimenti e calcoli.

\end{abstract}