\chapter{Strumenti e Tecnologie}
In questo capitolo vengono esaminati i principali strumenti e le tecnologie utilizzati durante lo svolgimento di tutte le attività, suddivisi in categorie
in base alla loro tipologia. Verranno trattati, con un approfondimento sul loro utilizzo pratico, gli ambienti di sviluppo integrati (IDE), i sistemi di
versionamento, i framework, le librerie e i linguaggi di programmazione, nonché le piattaforme per la gestione dei progetti e la collaborazione.

La scelta di queste tecnologie è stata influenzata principalmente dalle soluzioni già adottate da \textit{Peer Network} per la gestione e lo sviluppo dei progetti, 
ltre ad alcune proposte mirate a soddisfare esigenze operative specifiche.

Strumenti:
\begin{itemize}
    \item \textbf{Jira}: strumento di gestione progetti e monitoraggio sviluppato da Atlassian, principalmente per i team di sviluppo software. Questo strumento
    supporta metodologie Agile e consente di creare, assegnare e monitorare le attività. Tutte le attività svolte sono state registrate, organizzate e monitorate
    in un apposito progetto. Ogni progetto è strutturato con una gerarchia di epic, task e subtask, permettendo di suddividere il lavoro in unità più piccole e
    gestibili, in modo da mantenere un controllo chiaro e dettagliato sui progressi. Gli elementi della gerarchia sono stati costantemente aggiornati con dettagli
    e stato corrente, garantendo una visione sempre aggiornata. Le ore di lavoro dedicate a ciascun task e subtask sono state registrate tramite una funzionalità
    specifica e monitorate facilmente con l’applicazione integrata \textbf{Time Tracker Lite}, utilizzando filtri per l’analisi dei dati. Per pianificare le varie attività
    da svolgere, è stata utilizzata la dashboard \textbf{Timeline}, che permette di avere una panoramica chiara delle scadenze e delle priorità. Jira è stato integrato con
    Confluence per collegare attività e documentazione dello stesso progetto.
    \item \textbf{Confluence}: strumento di collaborazione e gestione della conoscenza sviluppato da Atlassian, progettato per centralizzare la creazione, l'organizzazione
    e la condivisione di documenti e pagine. È stato utilizzato per documentare i progetti, raccogliere appunti e facilitare la collaborazione in tempo reale tra i colleghi.
    Integrato con Jira, ha permesso di collegare facilmente la documentazione alle attività monitorate, mantenendo coerenza tra sviluppo e gestione operativa. Ogni progetto
    è organizzato in uno spazio dedicato, dove sono raccolti resoconti di riunioni, documentazione tecnica e materiali di supporto, garantendo un accesso rapido e
    strutturato alle informazioni. La \textit{Peer Network} utilizza questo strumento per creare e mantenere una knowledge base aziendale, utile come riferimento per risolvere
    problemi ricorrenti e condividere best practice. Grazie alla sua flessibilità, Confluence ha reso possibile aggiornare costantemente i contenuti e organizzare le
    informazioni in modo chiaro e accessibile a tutti i dipendenti.
    \item \textbf{Bitbucket}: piattaforma di gestione del codice sorgente sviluppata da Atlassian, è stata utilizzata per tracciare e gestire in modo efficiente le
    modifiche al codice tramite il controllo di versione \textbf{Git}. L'azienda ha repository dedicati per ogni progetto, implementando rigorosi controlli di accesso che
    arantiscono sia la sicurezza che una gestione centralizzata del codice. Questa piattaforma aiuta la collaborazione tra i membri del team di sviluppo, consentendo
    ciascun sviluppatore di contribuire al codice sorgente e monitorare in tempo reale le modifiche apportate dai colleghi.
    \item \textbf{Slack}: piattaforma di comunicazione e collaborazione aziendale progettata per ottimizzare il lavoro. Permette di inviare messaggi istantanei, effettuare
    chiamate vocali e video e condividere file in tempo reale. L’ambiente di lavoro \textit{Peer Network} è organizzato in canali tematici, suddivisi per progetto, reparto o argomento,
    facilitando la gestione delle conversazioni. Grazie all’integrazione con strumenti come Jira e Confluence, Slack supporta un flusso di lavoro collaborativo ed efficiente.
    Tutte le comunicazioni interne, sia tra colleghi che con l’azienda, sono state gestite tramite questa piattaforma, garantendo ordine e accessibilità alle informazioni.
    \item \textbf{Google Workspace}: \textit{Peer Network} adotta un dominio aziendale dedicato tramite Google, fornendo a ciascun dipendente un account personalizzato
    per accedere agli strumenti e alle risorse condivise. Il \textbf{Google Drive} aziendale è stato utilizzato per archiviare, consultare e modificare file e cartelle, con permessi
    di accesso configurati per garantire sicurezza e controllo. I \textbf{Google Calendar} personali sono stati condivisi tra i dipendenti, facilitando la pianificazione di riunioni
    e attività. Le riunioni sono state programmate su Calendar e svolte in modalità mista, consentendo ai dipendenti in smart working di partecipare tramite \textbf{Google Meet}.
    Per le presentazioni, sono stati utilizzati \textbf{Google Slides}, sia per le riunioni interne che per quelle con i clienti. \textbf{Gmail} ha centralizzato tutte le notifiche aziendali,
    incluse quelle provenienti da strumenti come Jira, Confluence, Bitbucket e Slack, offrendo un unico punto di accesso per il monitoraggio delle comunicazioni e delle attività.
    \item \textbf{Miro}: piattaforma di collaborazione online che ha semplificato la creazione e la condivisione in tempo reale di schemi e diagrammi, come la \textbf{\ac{WBS}},
    attraverso una lavagna virtuale accessibile a più colleghi simultaneamente. Miro è stato proposto all'azienda per sostituire l’uso di Google Slides, precedentemente impiegato
    per realizzare questi schemi senza uno strumento dedicato.
    \item \textbf{Visual Studio Code}: editor di codice sorgente usato come IDE durante lo sviluppo. Grazie all'integrazione con Git e all'uso di alcuni plugin, ha consentito
    di gestire facilmente le operazioni di controllo versione direttamente dall’interfaccia grafica, semplificando il controllo del codice sorgente.
    \item \textbf{MySQL Workbench}: software dedicato alla gestione di database MySQL. Grazie alla sua interfaccia, è stato possibile consultare e interrogare il database di \ac{PAM}
    tramite la creazione e l’esecuzione di query SQL.
\end{itemize}

Tecnologie:
\begin{itemize}
    \item \textbf{Vue.js}: framework JavaScript open-source progettato per la creazione di interfacce utente e applicazioni single-page. È noto per la sua reattività e
    facilità di integrazione con altre librerie. Basato su un’architettura a componenti, permette di organizzare il codice in moduli riutilizzabili e facilmente
    gestibili. Grazie ai suoi meccanismi reattivi, il Document Object Model (DOM) si aggiorna automaticamente al variare dei dati, supportando direttive dinamiche
    per la manipolazione del contenuto. Questo strumento è stato impiegato per lo sviluppo frontend di \ac{PAM}.
    \item \textbf{Vuex}: libreria di gestione dello stato per Vue.js, progettata per centralizzare e organizzare i dati in un unico store. Facilita la gestione di
    applicazioni complesse grazie a un flusso strutturato di mutazioni, che consente di modificare lo stato in modo tracciabile e prevedibile. È particolarmente utile
    per condividere dati tra componenti senza la necessità di passaggi manuali. Questa libreria è stata utilizzata per lo sviluppo frontend di \ac{PAM}.
    \item \textbf{Liferay}: piattaforma open-source progettata per la creazione di portali web e siti aziendali, offrendo soluzioni personalizzate per la gestione di
    contenuti, applicazioni e servizi online. La sua architettura modulare garantisce elevata personalizzazione e scalabilità, rendendola adatta a progetti complessi.
    Supporta diversi framework e tecnologie, tra cui JavaScript, favorendo l’integrazione con strumenti moderni di sviluppo. Questa piattaforma è stata utilizzata per
    gestire il frontend di \ac{PAM}, ospitando il codice sviluppato in Vue.js.
    \item \textbf{JavaScript}: linguaggio di programmazione utilizzato principalmente per lo sviluppo web. È eseguito direttamente nel browser e permette di creare
    pagine dinamiche e interattive. Viene utilizzato in Vue.js per creare e gestire i componenti e la logica dell'applicazione, oltre alla modifica dinamica del
    Document Object Model tramite direttive specifiche. Anche la gestione dello stato in Vuex è fatta in JavaScript.
    \item \textbf{MySQL}: sistema di gestione di database relazionali (RDBMS) utilizzato per archiviare, organizzare e recuperare dati in modo strutturato. Permette
    di interagire con i dati tramite il linguaggio SQL (Structured Query Language).
\end{itemize}